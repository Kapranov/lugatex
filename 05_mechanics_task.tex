\newpage	%--- page 5
\disableTemplate{LugatexLogo}
\disableTemplate{Navigatorbarbg}
\disableTemplate{Lugamathbg}
\AddToTemplate{Lugabgtest}
\AddToTemplate{Navigatorbarbg}
\AddToTemplate{LugatexLogo}

\subsection{Задачи с  примерами решения}
{\color{white}
\noindent
\textcolor{cyan}{\textbf{Задача}}
Найти функцию Лагранжа следующих систем, входящихся в однородном поле
тяжести, \textg\, --- ускорение силы тяжести.\\[5pt]
\noindent
\textcolor{magenta}{\textbf{Задание}}. Двойной плоский маятник,
внимательно исследуйте~\tooltip{рисунок}{23}.


\noindent
\textcolor{yellow}{\textbf{Решение}}. В качестве координат берем углы $\varphi_1$\, и $\varphi_2$,
которые нити $l_1$\, и $l_2$\, образуют с верикалью. Тогда для точки
$m_1$\, имеем:

\begin{center}
\begin{tikzpicture}
\tikzstyle{rec} = [rectangle,rounded corners,ultra thick,draw=monred!45]
\node[rec,text=white, drop shadow] {\parbox{0.6\textwidth}{%
\parbox{255pt}{%
$$
T_1 = \frac{1}{2} m_1 l^2_1\dot{\varphi}^2_i,\quad
U = - m_1\text{\textg}l_1 \cos\varphi_1
$$
}}};
\end{tikzpicture}
\end{center}

Чтобы найти кинетическую энергию второй точки, выражаем ее декартовы
координаты $x_2,\, y_2$\, (начало координат в точке подвеса, осью $y$\,
--- по вертикали вниз) через углы $\varphi_1\, \varphi_2$:
$$
x_2 = l_1 \sin \varphi_1 + l_2 \sin\varphi_2\qquad
y_2= l_1 \cos \varphi_1 + l_2 \cos \varphi_2
$$
После этого получим:
$$
T_2 = \frac{m_2}{2} \left(\dot{x}^2_2 + \dot{y}^2_2\right) =
\frac{m_2}{2}\left[
l^2_1\dot{\varphi}^2_i + l^2_2\dot{\varphi}^2_2 + 2l_1l_2\cos\left(
\varphi_1 - \varphi_2\right) \dot{\varphi}_1\dot{\varphi}_2
\right]
$$
\textcolor{yellow}{Окончательно}:\\[15pt]
\noindent
\begin{tikzpicture}
\filldraw[fill=yellow!50, draw=black!50, rounded corners=\boxroundness,drop
shadow={opacity=.84},shadow xshift=5pt, shadow yshift=7pt]
(0, 0) rectangle (410pt, 18pt);
\draw node[xshift=3pt,yshift=3pt,
inner sep=0pt,outer sep=0pt,anchor=south west] {%
{\color{black}
$
L = \frac{m_1 + m_2}{2} l^2_1\dot{\varphi}^2_1 + \frac{m_2}{2}
l^2_2\dot{\varphi}^2_2 + m_2l_1l_2 \dot{\varphi}_1 \dot{\varphi}_2\cos
\left(\varphi_1 - \varphi_2\right) + \left(m_1 +
m_2\right)\text{\textg}l_1\cos\varphi_1 + m_2\text{\textg}l_2\cos\varphi_2
$
}};
\end{tikzpicture}

\noindent
\textcolor{magenta}{\textbf{Задание}}. Плоский маятник с массой $m_2$,
точка подвеса которого (с массой $m_1$\, в ней) может совершать движение
по горизонтальной прямой, внимательно смотрите~\tooltip{рисунок}{24}.


\noindent
\textcolor{yellow}{\textbf{Решение}}. Вводя координату $x$\, точки
$m_1$\, и угол $\varphi$\, между нитью маятника и вертикалью,
получим:

\begin{center}
\begin{tikzpicture}
\filldraw[fill=yellow!50, draw=black!50, rounded corners=\boxroundness,drop
shadow={opacity=.84},shadow xshift=5pt, shadow yshift=7pt]
(0, 0) rectangle (236pt, 18pt);
\draw node[xshift=3pt,yshift=3pt,
inner sep=0pt,outer sep=0pt,anchor=south west] {%
{\color{black}
$
L = \frac{m_1 + m_2}{2}\dot{x}^2 + \frac{m_2}{2}
\left(l^2\dot{\varphi}^2 + 2l\dot{x}\dot{\varphi} \cos \varphi\right) +
m_2\text{\textg}l\cos\varphi
$
}};
\end{tikzpicture}
\end{center}

\vglue 5pt

\noindent
\textcolor{magenta}{\textbf{Задание}}. Плоский маятник,
точка подвеса которого:\\
\noindent
\textbf{a)}. равномерно движется по вертикальной окружности с постоянной частотой
$\gamma$\\ смотрите~\tooltip{рисунок}{25};\\
\noindent
\textbf{б)}. совершает горизонтальные колебания по закону $a \cos \gamma t$;\\
\noindent
\textbf{в)}. совершает вертикальные колебания по закону $a\cos \gamma t$.\\

\noindent
\textcolor{yellow}{\textbf{Решение}}.\\
\noindent
\textbf{а)} Координаты точки $m$:
$$
x = a \cos \gamma t + l \sin \varphi, \qquad
y = -a \sin \gamma t + l \cos \varphi
$$
Функция Лагранжа :
$$
L = \frac{ml^2}{2} \dot{\varphi}^2 + mla\gamma^2 \sin
\left(\varphi - \gamma t \right) + m\text{\textg}l\cos \varphi
$$
здесь опущены члены, зависящие только от времени, и исключена полная
прозводная по времени от $mal\gamma \cos\left(\varphi - \gamma t\right)$.\\

\noindent
\text{б)}. Координаты точки $m$:
$$
x = a \cos \gamma t + l \sin \varphi, \qquad y = l \cos\varphi
$$


Функция Лагранжа (после исключения полных производных):


$$
L = \frac{ml^2}{2} \dot{\varphi}^2 + mla\gamma^2 \cos \gamma t
\sin \varphi + m\text{\textg}l \cos \varphi
$$

\noindent
\textbf{в)}. Аналогичным образом:
$$
L = \frac{ml^2}{2}\dot{\varphi}^2 + mla\gamma^2 \cos\gamma t
\cos\varphi + m\text{\textg}l\cos\varphi
$$

\noindent
\textcolor{magenta}{\textbf{Задание}}. Система, изображенна
на~\tooltip{рисунок}{26}\, точка $m_2$\, движется по вертикальной
оси, а вся система вращается с постоянной угловой скоростью $\Omega$\,
вокруг этой оси.\\

\noindent
\textcolor{yellow}{\textbf{Решение}}. Вводим угол $\theta$\, между
отрезком $a$\, и вертикалью и угол поворота $\varphi$\, всей системы
вокруг оси вращения; $\dot{\varphi}=\Omega$. Для каждой из точек $m_1$\,
элемент перемещения $dl^2_1 =a^2d\theta^2 + a^2\sin^2\theta d\varphi^2$.
Для точки $m_2$\, расстояние от точки подвеса $A$\, равно $2a\cos\theta$,
и поэтому $dl_2 = - 2a \sin \theta d\theta$. Функция Лагранжа:

\vglue 15pt

\begin{center}
\begin{tikzpicture}
\filldraw[fill=yellow!50, draw=black!50, rounded corners=\boxroundness,drop
shadow={opacity=.84},shadow xshift=5pt, shadow yshift=7pt]
(0, 0) rectangle (305pt, 25pt);
\draw node[xshift=3pt,yshift=3pt,
inner sep=0pt,outer sep=0pt,anchor=south west] {%
{\color{black}
$
L = m_1 a^2 \left(\dot{\theta}^2+\Omega^2\sin^2\theta\right) + 2m_2a^2
\sin^2\theta\cdot \dot{\theta}^2 + 2\text{\textg}a \left(m_1 + m_2\right)\cos\theta
$
}};
\end{tikzpicture}
\end{center}
}
